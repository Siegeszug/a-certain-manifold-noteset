\documentclass[12pt, twosided]{article}
\input{sdpreamble}
\graphicspath{{./img/}}

\begin{document}
\noindent \textbf{Math 285} \hfill \textbf{Professor Loring Tu} \\
\textbf{Scribed by: Kyle Dituro} \hfill \textbf{September 6\tht, 2023}\hrule
\vspace{.2in}

\section{\texorpdfstring{\(C^\infty\)}{TEXT} functions on \texorpdfstring{\(\R^n\)}{TEXT}}

\begin{df}
  Let \(U \subset \R^n\) be open, \(p \in U\). \(f: U \to \R\) is \(C^k\) at \(p\) if all partials of \(f\) of order \(\leq k\) exist and are continuous at \(p\). E.g. \(C^0\) is continuous, \(C^1\) is continuous w/ first partials also continuous.
\end{df}

For the sake of simplicity in this course, we care only about functions which are \(C^\infty\), in other words \(f: U \to \R\) is \(C^\infty\) at \(p \in U\) if \(f\) is \(C^k\) for all \(k = 0, 1, 2, \ldots\).

We also care about another type of function, namely the analytic ones.

\begin{df}
  A function \(f: U \to \R\) is \textbf{analytic} at \(p\) if \(f(x)\) is equal to its Taylor's series at \(p\) in a neighborhood of \(p\).
\end{df}

Importantly, an analytic function is infinitely differentiable within its radius of convergence, since the power series can be differentiated term by term. This suggests the following lemma:

\begin{lm}
  Analytic \(\Rightarrow C^\infty\).
\end{lm}

Notice that, notably, the converse is not true:

\begin{exa}
  Take \(f(x) =
  \begin{cases}
    \frac{1}{e^{\frac{1}{x}}} & x > 0 \\
    0 & x \leq 0
  \end{cases}
  \). It is evidently clear that this function is \(C^\infty\). Derivative can easily be computed, and verifying that they are continuous is easy. However, notice that \(f^{(k)}(0) = 0\ \forall k = 0, 1, 2, \ldots\). Actually demonstrating this is a simple matter of induction. Therefore, the Taylor's series of \(f\) at \(x = 0\) is the zero function.

  Thus \(f(x)\) is not equal to its Taylor's series at \(0\) in any neighborhood of \(0\).

  Therefore, \(f\) is not analytic at zero.
\end{exa}

Luckily, for \(C^\infty\) functions there is an analogue that works.

The \textbf{Taylor's series with Remainder} of a function is equal to

\begin{align*}
  f(x) &= f(p) + \sum_{i = 1}^n \frac{\partial f}{\partial x^i}(p) (x^i - p^i) + \frac{1}{2!} \sum_{i, j} \frac{\partial^2 f}{\partial x^i \partial x^j}(p) (x^i - p^i)(x^j - p^j) + \ldots + k\th \text{ term } + \underbrace{k+1}_{\text{remainder}} \\
  p &= (p^1, \ldots, p^n) \in \R^n
\end{align*}

You know what a star-shaped polygon is. It's a polygon with a nonempty visibility kernel.

\begin{thm}{Taylor's Theorem with Remainder}
  Let \(U\) be a star-shaped open set in \(\R^n\) with respect to some point \(p \in U\).

  If \(f: U \to \R \) is \(C^\infty\), then \(\exists C^\infty\) functions \(g_1(x), \ldots, g_n(x)\) such that
  \begin{align*}
    f(x) = f(p) + \sum_{i = 1}^n g_i(x)(x^i - p^i) \ \text{and}\ g_i(p) = \frac{\partial f}{\partial x^i}(p).
  \end{align*}
\end{thm}

Now we remember calc III... uh oh.

\begin{center}
  \begin{tikzcd}
    & f \arrow[dl, dash] \arrow[d, dash] \arrow[dr, dash] & \\
    x^1 \arrow[rr, dotted, dash] \arrow[dr, dash]& \hspace{0pt} \arrow[d, dash] & x^n \arrow[dl, dash] \\
    & t & 
  \end{tikzcd}
\end{center}
\(\frac{d f}{dt} = \sum_{i = 1}^n \frac{\partial f}{\partial x^i} * \frac{d x^i}{d t}\).
Yeah, it's the chain rule.

\begin{proof}
  Let \(y \in U\) The parametrization of \(\overline{py}\) is \(x(t) = p + t(y - p)\).

  Note that \(x(0) = p\), \(x(1) = y\), \(x^i(t) = p^i + t(y^i - p^i) \frac{d x^i}{d t} = y^i - p ^i\).

  And also see
  \begin{align*}
    f(y) - f(p) &= f(x(1)) - f(x(0)) \\
                &= \int_0^1 \frac{d}{dt}f(x(t)) dt \\
                &= \int_0^1 \sum_i \frac{\partial f}{\partial x^i}(x(t)) \frac{dx^i}{dt} dt \\
                &= \sum_i \underbrace{\int_0^1 \frac{\partial f}{\partial x^i} (p + t(y - p)) dt * (y^i - p^i)}_{g_i(y)}
  \end{align*}

  And et cetera.
\end{proof}

\section{Tangent Vectors}

We have the issue where our current understanding of a tangent vector where our parameterization of the vectors are firmly rooted in \(\R^3\). So we're going to re-define the tangent vector by what we want it to do: compute the \textbf{direcitonal derivative}.

\begin{idea}
  If \(v_p \in T_p(U)\) and \(f \in C^\infty (U)\), then
  \begin{align*}
    D_{v_p} &= \text{directional derivative of } f \text{ in direction of} v_p \text{ at } p \\
            &= \left.\frac{d}{dt}\right\vert_{t=0} f(\underbrace{p + tv_p}_{\text{line thru } p \text{ with dir. } v_p}) \\
  \end{align*}
  Then, through application of the chain rule, we get
  \begin{align}
    D_{v_p} = \sum v^i \left.\frac{\partial}{\partial x^i}\right\vert_{p}
  \end{align}
\end{idea}

\begin{df}
  Let \((f, u)\) denote a \(C^\infty\) function \(f: U \to \R\), \(p \in U\). We say that \((f, U) \sim (g, V)\) if \(\exists W \subset U \cap V\) such that \(f = g\) on \(W\).
\end{df}

\begin{notn}
  \([(f, U)] = \) equiv class of \((f, U) = \) germ of \(f\) at \(p\).

  \(C_p^\infty = \opn \text{germs of } C^\infty\ \text{func at } p\cls\).
\end{notn}

\begin{lm}
  Germs are \textbf{module-like}, and we will call them a \(C_p^\infty\) algebra over \(\R\).
\end{lm}

\begin{prop}
  \(D_{v_p}: C_p^\infty \to \R\).
  \begin{enumerate}
  \item [(i)] \(D_{v_p}\)is \(\R\)-linear.
    \begin{align*}
      D_{v_p}(f + g) = D_{v_p}f + D_{v_p}g \\
      D_{v_p}(\lambda f) = \lambda D_{v_p}f.
    \end{align*}
  \item [(ii)] \(D_{v_p}(fg) = (D_{v_p}f) g(p) + f(p)D_{v_p}g\) (Liebniz rule)
  \end{enumerate}
\end{prop}
\begin{df}
  A function \(D = C_p^\infty \to \R\) satisfying conditions \((i)\) and \((ii)\) prior is called a \textbf{derivation at \(p\)} or a \textbf{point derivation} of \(C_p^\infty\).
\end{df}
\begin{df}
  \(\mathcal{D}_p(\R^n) = \opn \text{all point-derivations of } C_p^\infty\cls\)
\end{df}

Notice that the structure of \(\mathcal{D}_p(\R^n)\) has an addition and scalar multiplication structure, but notably not a vector multiplication structure. So this is a vector space.

\begin{thm}
  The map \(
  \begin{matrix}
    \phi : &T_p(\R^n) \to \mathcal{D}_p(\R^n) \\
           &v_p \mapsto D_{v_p}
  \end{matrix}\) is a linear isomorphism of vector spaces.
\end{thm}
\begin{lm}
  If \(D \in \mathcal{D}_p(\R^n)\) then \(Dc = 0\) \(\forall c \in \R\).
\end{lm}
\begin{proof}
  \(D1 = D(1 \cdot 1) = (D1) \cdot 1 + 1 \cdot D 1\), so \(0 = D1\). Thus \(Dc = D(c \cdot 1) = cD1 = c \cdot 0 = 0\).

  Thus
  \begin{align*}
    Df &= D(f(p) + \sum g_i(x)(x^i - p^i)) \\
       &= D(f(p)) + \sum (Dg_i)(p^i - p^i) + g_i(p D(x^i - p^i)) \\
       &= \sum (Dx^i)\left.\frac{\partial}{\partial x^i}\right\vert_p f \\
    &= \sum (Dx^i)e_{ip}
  \end{align*}
  Thus we have established the one-to-one correspondane
  \begin{align}
    v_p = \sum v^ie_{i,p} \longleftrightarrow D_{v_p} = \sum v^i \left.\frac{\partial}{\partial x^i}\right\vert_{p}
  \end{align}
\end{proof}
\begin{proof}[Of theorem] We do the usual thing.
  \begin{enumerate}
  \item[\textbf{Inj.}] Suppose \(\phi(v_p) = D = \sum v^i \left.\frac{\partial}{\partial x^i}\right\vert_p = 0\).
    Now apply to \(x^j\): \(\sum v^i \left.\frac{\partial}{\partial x^i}\right\vert_p x^j = 0\).
    And
    \begin{align*}
      \frac{\partial x^j}{\partial x^i} &=
      \begin{cases}
        1 & \text{for } j = i, \\
        0 & \text{for } j \neq i
      \end{cases}. \\
      = \delta_i^{\gamma}. \\
      \sum_i v^i \frac{\partial x^j}{\partial x^i}(p) &= \sum_i v^i \delta_i^j \\ &= v^j \forall j.
    \end{align*}
    So \(v_p = 0\).
  \item [\textbf{Surj.}] Let \(D \in \mathcal{D}_p(\R^n)\). Suppose \(D = D_{v_p}\) for \(v_p \in T_p(\R^n)\). So then

    \begin{align*}
      D = \sum_i v^i \left.\frac{\partial}{\partial x^i}\right\vert_p.
    \end{align*}

    Apply both sides to \(x^j\):
    \begin{align*}
      Dx^j = \sum v^i \left.\frac{\partial}{\partial x^i}\right\vert_p x^j = \sum_i v^i \delta^j_i = v^j.
    \end{align*}
    So \(v_p = \sum (Dx^i)\left.\frac{\partial}{\partial x^i}\right\vert_p\). then by Taylor's Theorem with Remainder , \(\exists g_i \in C^\infty(U)\) such that \(f(x) = f(p) + \sum g_i(x) (x^i - p^i)\), and \(g_i(p) = \frac{\partial f}{\partial x^i}(p)\).
  \end{enumerate}
\end{proof}

\begin{df}
  \(T_p(U) = \opn \text{point-derivations of } C_p^\infty\cls\)
\end{df}

\begin{df}
  A \textbf{vector field} on open \(U \subset \R^n\) \(X: U \to \coprod_{p \in U}T_p(U)\) (where \(\coprod\) denotes the disjoint union) such that \(X_p \in T_p(U)\).
\end{df}
\begin{note}
  If \(X_p \in T_p(U)\), then \(X_p = \sum a^i(p)\left.\frac{\partial}{\partial x^i}\right\vert_p\).
\end{note}

\begin{lm}
  \(X = \sum a6i \frac{\partial}{\partial x_i}\), where \(a^i: U \to \R\) are functions.
\end{lm}
\begin{df}
  A vector field \(X = \sum a^i \frac{\partial}{\partial x^i}\) on \(U\) is \(C^\infty\) if all \(a^i:U \to \R\) are \(C^\infty\) functions.
  \end{df}
  \begin{notn}
    \(\mathcal{X}(U) = \opn C^\infty \text{ v.f. on U }\cls\) is a vector space on \(\R\).
  \end{notn}
\end{document}
%%% Local Variables:
%%% mode: latex
%%% TeX-master: t
%%% End:
