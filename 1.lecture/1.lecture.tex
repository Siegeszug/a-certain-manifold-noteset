\documentclass[12pt, twosided]{article}
\usepackage[letterpaper,bindingoffset=0in,%
            left=1in,right=1in,top=1in,bottom=1in,%
            footskip=.25in]{geometry}

\usepackage{mathtools}
\usepackage{graphicx}

\usepackage{setspace}
\setstretch{1.1}

\usepackage{amsmath}
\usepackage{amsfonts}
\usepackage{amsthm}
\usepackage{amssymb}
\usepackage{csquotes}
\usepackage{relsize}

\usepackage{tikz}
\usetikzlibrary{cd}
\usetikzlibrary{fit,shapes.geometric}
\tikzset{%  
    mdot/.style={draw, circle, fill=black},
    mset/.style={draw, ellipse, very thick},
}

\usepackage{hhline}
\usepackage{systeme}
\usepackage{mathrsfs}
\usepackage{hyperref}
\usepackage{mathtools}  
\usepackage{silence}
\usepackage{blkarray}
\usepackage{float}
\usepackage{framed}
\usepackage{array}
\usepackage{stmaryrd}
\usepackage{extarrows}
\usepackage{caption}
\captionsetup[figure]{labelfont={bf},name={Fig.},labelsep=period}

\theoremstyle{definition}
\newtheorem{df}{Definition}
\newtheorem{exa}{Example}
\newtheorem{ques}{Question}
\newtheorem{exr}{Exercise}
\newtheorem{prb}{Problem}
\newtheorem{notn}{Notation}
\newtheorem*{note}{Note}
\theoremstyle{plain}
\newtheorem{thm}{Theorem}
\newtheorem{prop}{Proposition}
\newtheorem{conj}{Conjecture}
\newtheorem{cor}{Corollary}
\newtheorem{lm}{Lemma}
\newtheorem*{fact}{Fact}
\newtheorem*{idea}{Idea}
\newtheorem*{clm}{Claim}
\newtheorem*{rmk}{Remark}
\usepackage[ruled]{algorithm2e}

\usepackage{ulem}
\makeatletter

\def\lf{\left\lfloor}   
\def\rf{\right\rfloor}
\def\lc{\left\lceil}   
\def\rc{\right\rceil}
\def\st{\text{ s.t. }}
\def\1{^{-1}}
\def\ind{\mathbf{1}}
\def\R{\mathbb{R}}
\def\Q{\mathbb{Q}}
\def\Z{\mathbb{Z}}
\def\C{\mathbb{C}}
\def\I{\mathbb{I}}
\def\N{\mathbb{N}}
\def\F{\mathbb{F}}
\def\A{\mathbb{A}}
\def\Li{\text{Li}}
\def\th{^\text{th}}
\def\sp{\text{Sp}}
\def\opn{\left\{}
\def\cls{\right\}}
\def\Aut{\text{Aut}}
\def\PG{\text{PG}}
\def\GL{\text{GL}}
\def\PGL{\text{PGL}}
\def\Cov{\text{Cov}}
\def\Pack{\text{Pack}}
\def\PgamL{\text{P}\Gamma\text{L}}
\def\gamL{\Gamma\text{L}}
\def\cl{\text{cl}}
\def\stbar{\ \middle\vert\ }
\def\partdone{\hphantom{1} \hfill \(\triangle\)}
\def\s0{_0}
\def\s1{_1}
\def\s2{_2}
\def\id{\mathrm{id}}
\def\topn{\text{ open}}
\def\Bd{\text{Bd }}
\def\nope{\(\longrightarrow\!\!\longleftarrow\)}
\def\stt{\(^{\text{st}}\ \)}
\def\tht{\(^{\text{th}}\ \)}
\def\ndt{\(^{\text{nd}}\ \)}
\renewcommand{\P}{\mathbb{P}}
\newcommand{\leg}[2]{\left( \frac{#1}{#2} \right)}

\renewcommand*\env@matrix[1][*\c@MaxMatrixCols c]{%
   \hskip -\arraycolsep
   \let\@ifnextchar\new@ifnextchar
   \array{#1}}
\makeatother

% These two lines suppress the warning generated 
% by amsmath for overwriting the choose command  
% because it's annoying. This probably has unint-
% ended ramifications somewhere else, but I'm too
% lazy to actually figure that out, so we'll cro-
% ss that bridge when we come to it lol.
\renewcommand{\choose}[2]{\left( {#1 \atop #2} \right)}
\WarningFilter{amsmath}{Foreign command} 

\renewcommand{\mod}[1]{\ (\mathrm{mod}\ #1)}
\renewcommand{\vec}[1]{\mathbf{#1}}

\let\oldprime\prime
\def\prime{^\oldprime}

\usepackage{float}
\restylefloat{figure}

\usepackage{cleveref}
\Crefname{thm}{Theorem}{Theorems}

% Comment commands for co-authors
\newcommand{\kmd}[1]{{\color{purple} #1}}

\newcolumntype{L}{>{$}l<{$}}
% Bib matter
\let\oldepsilon\epsilon
\def\epsilon{\varepsilon}

\let\oldphi\phi
\def\phi{\varphi}

%%% Local Variables:
%%% mode: plain-tex
%%% TeX-master: t
%%% End:

\graphicspath{{./img/}}

\begin{document}
\noindent \textbf{Math 285} \hfill \textbf{Professor Loring Tu} \\
\textbf{Scribed by: Kyle Dituro} \hfill \textbf{September 6\tht, 2023}\hrule
\vspace{.2in}

\section{\texorpdfstring{\(C^\infty\)}{TEXT} functions on \texorpdfstring{\(\R^n\)}{TEXT}}

\begin{df}
  Let \(U \subset \R^n\) be open, \(p \in U\). \(f: U \to \R\) is \(C^k\) at \(p\) if all partials of \(f\) of order \(\leq k\) exist and are continuous at \(p\). E.g. \(C^0\) is continuous, \(C^1\) is continuous w/ first partials also continuous.
\end{df}

For the sake of simplicity in this course, we care only about functions which are \(C^\infty\), in other words \(f: U \to \R\) is \(C^\infty\) at \(p \in U\) if \(f\) is \(C^k\) for all \(k = 0, 1, 2, \ldots\).

We also care about another type of function, namely the analytic ones.

\begin{df}
  A function \(f: U \to \R\) is \textbf{analytic} at \(p\) if \(f(x)\) is equal to its Taylor's series at \(p\) in a neighborhood of \(p\).
\end{df}

Importantly, an analytic function is infinitely differentiable within its radius of convergence, since the power series can be differentiated term by term. This suggests the following lemma:

\begin{lm}
  Analytic \(\Rightarrow C^\infty\).
\end{lm}

Notice that, notably, the converse is not true:

\begin{exa}
  Take \(f(x) =
  \begin{cases}
    \frac{1}{e^{\frac{1}{x}}} & x > 0 \\
    0 & x \leq 0
  \end{cases}
  \). It is evidently clear that this function is \(C^\infty\). Derivative can easily be computed, and verifying that they are continuous is easy. However, notice that \(f^{(k)}(0) = 0\ \forall k = 0, 1, 2, \ldots\). Actually demonstrating this is a simple matter of induction. Therefore, the Taylor's series of \(f\) at \(x = 0\) is the zero function.

  Thus \(f(x)\) is not equal to its Taylor's series at \(0\) in any neighborhood of \(0\).

  Therefore, \(f\) is not analytic at zero.
\end{exa}

Luckily, for \(C^\infty\) functions there is an analogue that works.

The \textbf{Taylor's series with Remainder} of a function is equal to

\begin{align*}
  f(x) &= f(p) + \sum_{i = 1}^n \frac{\partial f}{\partial x^i}(p) (x^i - p^i) + \frac{1}{2!} \sum_{i, j} \frac{\partial^2 f}{\partial x^i \partial x^j}(p) (x^i - p^i)(x^j - p^j) + \ldots + k\th \text{ term } + \underbrace{k+1}_{\text{remainder}} \\
  p &= (p^1, \ldots, p^n) \in \R^n
\end{align*}

You know what a star-shaped polygon is. It's a polygon with a nonempty visibility kernel.

\begin{thm}{Taylor's Theorem with Remainder}
  Let \(U\) be a star-shaped open set in \(\R^n\) with respect to some point \(p \in U\).

  If \(f: U \to \R \) is \(C^\infty\), then \(\exists C^\infty\) functions \(g_1(x), \ldots, g_n(x)\) such that
  \begin{align*}
    f(x) = f(p) + \sum_{i = 1}^n g_i(x)(^i - p^i) \ \text{and}\ g_i(p) = \frac{\partial f}{\partial x^i}(p).
  \end{align*}
\end{thm}

Now we remember calc III... uh oh.

\begin{center}
  \begin{tikzcd}
    & f \arrow[dl, dash] \arrow[d, dash] \arrow[dr, dash] & \\
    x^1 \arrow[rr, dotted, dash] \arrow[dr, dash]& \hspace{0pt} \arrow[d, dash] & x^n \arrow[dl, dash] \\
    & t & 
  \end{tikzcd}
\end{center}
\(\frac{d f}{dt} = \sum_{i = 1}^n \frac{\partial f}{\partial x^i} * \frac{d x^i}{d t}\).
Yeah, it's the chain rule.

\begin{proof}
  Let \(y \in U\) The parametrization of \(\overline{py}\) is \(x(t) = p + t(y - p)\).

  Note that \(x(0) = p\), \(x(1) = y\), \(x^i(t) = p^i + t(y^i - p^i) \frac{d x^i}{d t} = y^i - p ^i\).

  And also see
  \begin{align*}
    f(y) - f(p) &= f(x(1)) - f(x(0)) \\
                &= \int_0^1 \frac{d}{dt}f(x(t)) dt \\
                &= \int_0^1 \sum_i \frac{\partial f}{\partial x^i}(x(t)) \frac{dx^i}{dt} dt \\
                &= \sum_i \underbrace{\int_0^1 \frac{\partial f}{\partial x^i} (p + t(y - p)) dt * (y^i - p^i)}_{g_i(y)}
  \end{align*}

  And et cetera.
\end{proof}
\end{document}
%%% Local Variables:
%%% mode: latex
%%% TeX-master: t
%%% End:
